Proposed Timeline and Feasibility Assessment 
This project embarks on a comprehensive exploration of parenting norms among U.S. college students, structured over a carefully planned timeline to ensure both methodological rigor and practical feasibility. The initial phase of the project, spanning one month, is devoted to analyzing historical data on parenting norms from the American Time Use Survey for the years 2003 to 2022. This crucial step establishes a reliable baseline for understanding changes in parenting norms over time, utilizing historical data analysis techniques validated in prior research to lay a solid foundation for the study (Craig, 2011; Dotti Sani & Treas, 2016; Gimenez-Nadal & Molina, 2013). Following this, the project allocates four months to the revision and administration of a survey questionnaire to a targeted sample of 1000 to 1500 U.S. college students. This sample size is chosen based on considerations of convenience, resource constraints, and the ease of obtaining Institutional Review Board (IRB) approval, a common prerequisite for research involving human subjects in academic settings. The synergy between the analysis of historical data and contemporary survey methods mirrors best practices in social science research, enhancing the project's methodological soundness (Kilgallen et al., 2021; Lawson et al., 2021b).
Subsequently, the next five months are dedicated to the development, testing, and refinement of an Agent-Based Modeling (ABM) simulation. This simulation, informed by socio-economic and demographic data from the student surveys, aims to model the dissemination and adoption of parenting norms. ABM's utility in simulating complex social interactions and norm transmission has been well-documented in similar investigations, making it a pivotal tool for uncovering the dynamics behind parenting norm spread among college students(Conte & Paolucci, 2014; Liang et al., 2022; Zia et al., 2019). The project's final phase, lasting four months, focuses on data analysis, the dissemination of findings, and preparation for publication. This timeline aligns with the standard processes of similar studies, providing sufficient time for a thorough analysis and the peer review process. 
The project's design draws on methodologies that have been effectively applied in related fields, enhancing its feasibility and potential for significant contributions to the understanding of parenting norms. By addressing potential challenges, such as the limitations inherent in self-reported data and ensuring a robust comparison of perceived norms against established baselines, the project is well-positioned to offer insightful findings. Moreover, the inclusion of various social learning strategies in the ABM simulations enriches the analysis, offering a nuanced understanding of the factors influencing the spread of parenting norms among U.S. college students. This comprehensive approach, coupled with the project's methodological rigor and strategic planning, underscores its potential to advance the field of social science research on parenting norms.
