% If you want to compile this section only, make sure to include relevant document headers and the \being \end document commands.
% You can make this a bit easier if you use the subfile package
Preliminary Results
Figure 2 traces the evolution of parental time investment in childcare from 2003 to 2021, drawing on data from the American Time Use Survey to offer a quantifiable representation of parenting norms. The graph reveals a gradual upward trajectory in the average minutes parents dedicate to childcare per day, with the line indicating a notable increase from 2003's average of approximately 65 minutes to over 75 minutes by 2021. This trend underscores a shift towards more time-intensive parenting practices, reflecting broader societal movements towards 'intensive parenting,' where the quality and quantity of time spent with children are increasingly valued for their developmental benefits. The highlighted average of 70 minutes over the period serves as a baseline against which the year-to-year fluctuations can be assessed. These variations in time investment may correlate with shifts in cultural expectations, economic conditions, and changes in family structure that have influenced parenting approaches over nearly two decades. This longitudinal analysis forms the basis for investigating the alignment (or misalignment) between actual parental behaviors and societal perceptions of parenting norms among U.S. college students, exploring the potential misperceptions that may exist about the extent and nature of parental involvement in childcare. 
 
Figure 2. Trend of Time Spent on Childcare Activities

By default, our ABM model is configured using real-world data for initial time investment, with the assumption of high education level parents aligning closely with the optimal time value. This setup reflects empirical findings, showing that high education level parents not only invest more optimal time in their children but also have a higher individual learning ratio and fewer connections with their surroundings compared to parents from lower education levels. The network graph on the left IN Figure 3 illustrates these interactions, with node colors indicating education levels (darkslategray for high, cadetblue for medium, and paleturquoise for low) and node sizes representing time investments. The right panels track the progression of average child outcome scores and the number of social learning agents across different education levels over 50 steps.
The top right chart in Figure 3 illustrates that high education level parents quickly achieve optimal child outcome scores, stabilizing at the maximum score of 20 early in the simulation. In contrast, medium and low education level parents gradually improve their child outcome scores, taking longer to reach stability. Meanwhile, the bottom right chart shows the dynamic shift in the number of social learning agents over time. High education level parents maintain a low number of social learning agents, reflecting their higher reliance on individual learning. In contrast, medium and low education level parents exhibit more significant fluctuations in social learning agents, indicating a greater reliance on social learning strategies.
 
Figure 3. Simulation Results under Default Settings

Evaluation
The preliminary results demonstrate that the project's innovative methodology, combining empirical data collection with agent-based modeling, has successfully addressed its primary objectives of exploring the dynamics of parenting norms among U.S. parents and college students and the potential misperceptions therein. Specifically, the trend of time spent on childcare activities, as depicted in Figure 2, provides a tangible baseline of actual parenting norms, which is crucial for assessing the effectiveness of the ABM in simulating the spread and adoption of these norms within the student population. The ABM simulations, detailed in Figure 3 further elucidate how different learning strategies—Frequency Dependent Learning, Success Base Learning, and Random Copying—affect the dissemination and adoption of intensive parenting norms, offering insights into the mechanisms through which perceptions and misperceptions of norms are formed and altered.
However, our model incorporates several simplifying assumptions that may limit its ability to fully capture the complexities of real-world parenting norms. First, the model includes only one type of agent, representing parents, and does not consider the roles of other influential actors such as family members, educators, or peers. These additional agents play crucial roles in shaping child outcomes and parental strategies. Their absence in the model limits its comprehensiveness and fails to account for the broader social dynamics that influence parenting practices.
Furthermore, the setup for the social networks in the model is less than ideal. Due to the restrictions of the Mesa library, the social networks of different education levels are interconnected. This interconnectedness does not accurately mimic real-world scenarios, where social barriers often exist between parents of different socioeconomic statuses. These barriers can significantly influence social learning and the adoption of parenting practices. The model's inability to represent these social divisions may limit its ability to accurately simulate the spread and influence of intensive parenting practices within distinct social groups.
