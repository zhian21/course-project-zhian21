\documentclass{article}

\title{Big Picture Draft}
\author{Andy Chen}
\date{04/21/2024}

\begin{document}


\maketitle
 


\section{Introduction \& Literature Review}

\subsection*{Context}
Social learning plays a pivotal role in the dissemination of social norms, significantly influencing individual behaviors and the collective practices that define human societies. However, when misperceptions occur, they can lead to behaviors that are misaligned with actual norms, prompting the use of social norm approaches to recalibrate individual perceptions towards reality, particularly in areas like public health.


\subsection*{Your Research Question}
This project focuses on parenting norms as a case study of social norms, aiming to investigate the dissemination of misperceptions through various social learning strategies. By examining how misperceptions about parenting norms are formed and sustained within social networks, the project seeks to understand the mechanisms behind the spread and correction of these misperceptions.

\subsection*{What does the existing literature say}
The existing body of literature highlights the prevalence of misperceptions across a wide range of behaviors and populations, underscoring the complex nature of these phenomena. Efforts aimed at correcting these misperceptions have yielded mixed outcomes in terms of behavior compliance, emphasizing the intricate process involved in altering social norms. This situation underscores an urgent need for further research to deepen our understanding of these dynamics. A particularly notable gap in current approaches is their ineffectiveness, which can largely be attributed to a limited grasp of the underlying mechanisms that drive the spread of misperceptions.

\subsection*{Significance with respect to existing knowledge}
This project seeks to bridge a significant gap in existing research by exploring how misperceptions in parenting norms are spread and corrected, thereby facilitating the alignment of behaviors with actual social norms. It aims to improve social learning processes and guide the creation of interventions that accurately address and rectify misperceived norms, fostering healthier social and familial environments.

\section{Data and Methods}
\subsection*{State data and justify}
Our research employs an Agent-Based Modeling (ABM) approach, supported by surveys and social media analysis, to dissect the formation and correction of misperceptions in parenting norms among college students. This design is chosen for its capacity to simulate the intricate social interactions that lead to the spread of misperceptions, allowing for a detailed examination of how individual beliefs and societal expectations regarding parenting interplay (). By integrating direct survey data with dynamic social media insights, our methodology offers a comprehensive lens through which the transmission and evolution of parenting norms can be observed and understood, providing a solid basis for developing targeted normative interventions.


\subsection*{State method and justify}
The analytical method employed in this project is Agent-Based Modeling (ABM), chosen for its unique ability to simulate complex social interactions and the dynamic spread of misperceptions among individuals. ABM facilitates a nuanced exploration of how personal beliefs and societal pressures regarding parenting converge, making it a critical  tool for uncovering the mechanisms behind the dissemination and correction of misperceived norms within social networks.


\section{Feasibility}
\subsection*{Evaluation of approach w.r.t. RQ/project goal}
The chosen approach of employing Agent-Based Modeling, supplemented by surveys and social media analysis, is well-suited to addressing the research question regarding the dissemination and correction of misperceptions in parenting norms among college students. This methodology aptly captures the complex social dynamics and interactions that contribute to the formation and sustenance of misperceptions, thereby aligning closely with the project's goal of understanding and rectifying these misperceptions to realign behaviors with actual social norms.

\subsection*{Initial Results (or Mock-up)}
If the project progresses as anticipated, the initial results are expected to display detailed simulations that map the formation, persistence, and rectification of misperceptions about parenting norms among college students, using Agent-Based Modeling. Specifically, these outcomes will be augmented by targeted survey responses and social media analysis, pinpointing the specific variances between students' perceived and actual parenting norms and uncovering strategic points for targeted normative interventions.

\subsection*{Proposed timeline}
The project is structured to unfold over a 12-month period, beginning with the initial three months dedicated to collecting survey data and social media analysis, followed by six months of developing and refining the Agent-Based Modeling simulations. The final three months will be allocated for analyzing the results, identifying intervention strategies, and preparing the findings for dissemination.

\subsection*{Securing an Advisor/Sponsor}
Potential advisors for this project include Professor Jean Clipperton, due to her expertise in computational social science, who can provide valuable guidance on the use of Agent-Based Modeling in social norms research. Additionally, the Association for Computational Social Science represents a promising sponsor, offering both financial support and academic resources necessary for the successful execution of the project, given its alignment with their interests in advancing computational methods to understand social behaviors.

\subsection*{Cost and funding (if applicable)}
The project's budget is estimated to require approximately $50,000 to cover the costs of conducting comprehensive surveys, detailed social media analysis, and the employment of research assistants for Agent-Based Modeling (ABM) simulations. This revised estimate includes $20,000 for survey incentives and platform fees, $15,000 for social media data acquisition and analysis tools, and $15,000 allocated towards the salaries of research assistants who will support the computational and analytical aspects of ABM.


\section*{Assessment of the overall structure and alignment}
The project's use of ABM, supplemented by surveys and social media analysis, aligns well with its goals but may face biases from digital data and oversimplification of behaviors. Despite these concerns, its multidisciplinary approach, supported by expert advisors and sponsors, provides a solid framework for exploring parenting norm misperceptions among college students.


\end{document}