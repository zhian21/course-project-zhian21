\documentclass{article}

\title{Literature Review}
\author{Andy Chen}
\date{04/21/2024}

\begin{document}


\maketitle


\subsection*{Social Learning in Humans}
Social learning, a fundamental mechanism of human adaptation and cultural evolution, involves the acquisition of behaviors, skills, and knowledge through observing and imitating others (Kendal et al., 2009). This process is essential for the transmission of cultural norms, practices, and information across generations, facilitating the survival and success of human societies. Social learning operates through various modalities, including direct instruction, observation, and imitation, allowing individuals to rapidly acquire complex behaviors that would be difficult or impossible to learn independently (Mesoudi, 2016). However, not all human behaviors can be effectively learned through simple observation. Complex behaviors, which require a deeper understanding of underlying principles or contexts, often necessitate more sophisticated forms of social learning and direct experience. This limitation points to the necessity of exploring additional mechanisms through which humans acquire complex social behaviors.

\subsection*{The Influence of Social Norms on Decision-Making}
Social norms, the implicit rules that dictate behavior within groups and societies, act as crucial guides for individuals, enabling them to navigate complex social behaviors without explicit instructions. These norms, deeply ingrained in the cultural and societal framework, inherently influence actions and decisions (Legros & Cislaghi, 2020). A notable study by Henrich et al. (2003), which involved economic experiments like the ultimatum game across fifteen small-scale societies, illustrates the significant role of social norms in shaping economic decision-making (Henrich & McElreath, 2003). The study found that decisions varied widely among participants from different societies, directly correlating with the prevailing social norms of their communities. This underscores the powerful influence of social norms on individual behaviors. Definitions of social norms, however, differ across various fields, including sociology, psychology, and anthropology, each adding its own perspective to the understanding of these norms (Chung & Rimal, 2016). For this study, social norms are defined as the collective expectations and rules that govern group behavior, emphasizing the importance of shared understanding and expectations in influencing behaviors, rather than relying solely on individual learning mechanisms.

\subsection*{The Impact of Misperceptions on Behavior}
A critical issue arising from the reliance on social norms for behavior learning is the phenomenon of misperceptions. Misperceptions occur when there is a discrepancy between individuals' perceptions of social norms and the actual behaviors and attitudes of their community. This misalignment can lead to individuals conforming to what they mistakenly believe to be the norm, perpetuating behaviors that may not reflect the true values or practices of the group (Bjerring et al., 2014; Miller & Mcfarland, 1987). The prevalence of misperceptions is a widespread phenomenon, documented in various contexts, including health behaviors, environmental practices, and social attitudes (Bursztyn et al., 2023; Mastroianni et al., 2022; Perkins, 2014). One specific case study involves the misperception around social distancing behaviors during the COVID-19 pandemic, where individuals underestimated the extent to which others were complying with social distancing guidelines, potentially leading to lower adherence to such guidelines themselves (de Mooij et al., 2022). These cases illustrate how misperceptions can significantly influence individual and collective behaviors, often leading to negative outcomes.

\subsection*{Addressing Misperceptions Through the Social Norm Approach}
In response to the challenges posed by misperceptions, the social norm approach has emerged as a strategy to correct or recalibrate individuals' misperceptions of social norms. This approach involves providing accurate information about the actual norms and behaviors within a community, aiming to align individuals' perceptions with reality (Berkowitz et al., 2022). Studies have demonstrated the effectiveness of this approach in various settings, including reducing alcohol consumption among college students and promoting sustainable environmental practices (Huber et al., 2018; Perkins, 2014). However, the social norm approach is not without limitations. Its effectiveness can be hindered by the complexity of social norms and the mechanisms through which misperceptions are spread and maintained (Dempsey et al., 2018). Understanding these underlying processes is crucial for enhancing the efficacy of interventions aimed at correcting misperceptions.

\subsection*{Utilizing Agent-Based Modeling to Understand the Dynamics of Social Norms}
Agent-based Modelling (ABM) is a computational modeling approach that simulates the actions and interactions of autonomous agents (individuals or collective entities such as organizations) to assess their effects on the system as a whole (Conte & Paolucci, 2014). ABM is particularly useful for studying the spread of norms via social interactions, as it allows researchers to explore how individual behaviors can lead to the emergence of collective phenomena (Pavón et al., 2008; Prskawetz, 2017). One application of ABM is in understanding how social norms influence health behaviors in a community. By simulating the interactions between individuals with varying perceptions and behaviors, researchers can identify the conditions under which certain health behaviors become normative and how misperceptions can be corrected through targeted interventions. This approach provides valuable insights into the dynamics of social norms and their impact on individual and collective behaviors, offering a powerful tool for designing more effective public health strategies.

\subsection*{Reference}
Berkowitz, A. D., Bogen, K. W., Meza Lopez, R. J., Mulla, M. M., & Orchowski, L. M. (2022). The social norms approach as a strategy to prevent violence perpetrated by men and boys: A review of the literature. In Engaging Boys and Men in Sexual Assault Prevention: Theory, Research, and Practice (pp. 149–181). Elsevier. https://doi.org/10.1016/B978-0-12-819202-3.00009-2

Bjerring, J. C., Hansen, J. U., & Pedersen, N. J. L. L. (2014). On the rationality of pluralistic ignorance. Synthese, 191(11), 2445–2470. https://doi.org/10.1007/s11229-014-0434-1

Bursztyn, L., Cappelen, A. W., Tungodde, B., Voena, A., Yanagizawa-Drott, D., Tungodden, B., Allocchio, C., Carvajal, D., Escobar, B., Kim, A., & Ziegler, C. (2023). How Are Gender Norms Perceived? How Are Gender Norms Perceived? *. https://ssrn.com/abstract=4394127

Chung, A., & Rimal, R. N. (2016). Social norms: A review. Review of Communication Research, 4, 1–28. https://doi.org/10.12840/issn.2255-4165.2016.04.01.008

Conte, R., & Paolucci, M. (2014). On agent-based modeling and computational social science. Frontiers in Psychology, 5(JUL). https://doi.org/10.3389/fpsyg.2014.00668

de Mooij, J., Dell’Anna, D., Bhattacharya, P., Dastani, M., Logan, B., & Swarup, S. (2022). Quantifying the Effects of Norms on COVID-19 Cases Using an Agent-Based Simulation. Lecture Notes in Computer Science (Including Subseries Lecture Notes in Artificial Intelligence and Lecture Notes in Bioinformatics), 13128 LNAI, 99–112. https://doi.org/10.1007/978-3-030-94548-0_8

Dempsey, R. C., McAlaney, J., & Bewick, B. M. (2018). A critical appraisal of the social norms approach as an interventional strategy for health-related behavior and attitude change. In Frontiers in Psychology (Vol. 9, Issue NOV). Frontiers Media S.A. https://doi.org/10.3389/fpsyg.2018.02180

Henrich, J., & McElreath, R. (2003). The Evolution of Cultural Evolution. In Evolutionary Anthropology (Vol. 12, Issue 3, pp. 123–135). https://doi.org/10.1002/evan.10110

Huber, R. A., Anderson, B., & Bernauer, T. (2018). Can social norm interventions promote voluntary pro environmental action? Environmental Science and Policy, 89, 231–246. https://doi.org/10.1016/j.envsci.2018.07.016

Kendal, J., Giraldeau, L. A., & Laland, K. (2009). The evolution of social learning rules: Payoff-biased and frequency-dependent biased transmission. Journal of Theoretical Biology, 260(2), 210–219. https://doi.org/10.1016/j.jtbi.2009.05.029

Legros, S., & Cislaghi, B. (2020). Mapping the Social-Norms Literature: An Overview of Reviews. Perspectives on Psychological Science, 15(1), 62–80. https://doi.org/10.1177/1745691619866455

Mastroianni, A. M., Dana, J., & Weber, E. (2022). Widespread misperceptions of long-term attitude change. https://doi.org/10.1073/pnas.2107260119/-/DCSupplemental

Mesoudi, A. (2016). Cultural evolution: Integrating psychology, evolution and culture. In Current Opinion in Psychology (Vol. 7, pp. 17–22). Elsevier. https://doi.org/10.1016/j.copsyc.2015.07.001

Miller, D. T., & Mcfarland, C. (1987). Pluralistic Ignorance: When Similarity is Interpreted as Dissimilarity. In Journal of Personality and Social Psychology (Vol. 53, Issue 2).

Pavón, J., Arroyo, M., Hassan, S., & Sansores, C. (2008). Agent-based modelling and simulation for the analysis of social patterns. Pattern Recognition Letters, 29(8), 1039–1048. https://doi.org/10.1016/j.patrec.2007.06.021

Perkins, W. (2014). Misperception Is Reality: The “Reign of Error” About Peer Risk Behaviour Norms Among Youth and Young Adults. In The Complexity of Social Norms                                                              . http://www.springer.com/series/11784

Prskawetz, A. (2017). The Role of Social Interactions in Demography: An Agent-Based Modelling Approach. In Springer Series on Demographic Methods and Population Analysis (Vol. 41, pp. 53–72). Springer Science and Business Media B.V. https://doi.org/10.1007/978-3-319-32283-4_3
 
\end{document}