% If you want to compile this section only, make sure to include relevant document headers and the \being \end document commands.
% You can make this a bit easier if you use the subfile package

Introduction 
The complexity of human social learning
Social learning, as a pivotal evolutionary mechanism, significantly underpins the development and dissemination of complex cultural practices, technologies, and social norms, thereby facilitating the rapid adaptation of populations to changing environments (Kendal et al., 2018; Morgan et al., 2012; Whiten & van de Waal, 2017). This process, termed cumulative cultural evolution, is instrumental in the high-fidelity transmission of cultural knowledge across generations, markedly contributing to human ecological success(Henrich et al., 2016; Henrich & McElreath, 2003). It fosters the preservation and accumulation of cultural knowledge beyond the capabilities of individual learning, profoundly influencing a broad spectrum of daily activities and molding the intricate web of social and cultural practices that define human societies.
Despite its critical role, it is imperative to recognize that not all behaviors can be directly observed and internalized, presenting a notable challenge for accurate comprehension and assimilation of such behaviors. This underscores the complexity of social learning processes, particularly in the acquisition of social norms (Chung & Rimal, 2016; Legros & Cislaghi, 2020). According to Mackie et al. (2014), a social norm is constructed from the beliefs and desires of individuals within a reference group, where conformity to the norm is driven by empirical expectations (beliefs about what others do) and normative expectations (beliefs about what others think one should do). An illustrative example of this definition is evident in a case study on energy conservation in households. In this study, social norms were leveraged to influence behavior by disseminating information about the energy consumption patterns of neighbors. The researchers observed that households reduced their energy consumption when they believed that their neighbors were also engaging in energy-saving behaviors, thereby aligning with the previous framework of empirical and normative expectations (Pavón et al., 2008).
These norms, primarily learned through observation and social interaction, are susceptible to misperception, a phenomenon where there is a discrepancy between the perceived and the actual behaviors and attitudes within a community, due to their inherent subtlety and the implicit nature of their transmission(Bursztyn & Yang, 2022; Mastroianni et al., 2022). Such misperceptions are further exacerbated in situations where direct observation is impeded, leading individuals to rely on second-hand information or biased narratives, thus magnifying the vulnerability of social norms to misperception (Bjerring et al., 2014; Miller & Mcfarland, 1987). This delineates the intricate challenges faced in accurately transmitting and adhering to social norms within human societies, highlighting the nuanced complexities inherent in the social learning processes.
Indeed, misperceptions are a widespread phenomenon across various domains, including politics, socioeconomics, education, and gender norms, reflecting genuine beliefs that significantly deviate from reality (Bursztyn & Yang, 2022). This prevalence is not merely a result of measurement errors but indicates a deep-rooted challenge in aligning individual beliefs with factual information (Lawson et al., 2021a). For example, in the context of gender norms, research has provided insights into the misperceived social norms regarding women's employment outside the home in Saudi Arabia. This study revealed a significant discrepancy: a majority of men privately support the notion of women working outside the home but substantially underestimate societal support for this idea (Bursztyn et al., 2020). This finding not only emphasizes the complexity of misperceptions but also illustrates how they can both influence and be influenced by social norms, cultural contexts, and individual experiences. Similarly, in the health domain, during the COVID-19 pandemic, a specific case study highlighted how misperceptions could distort understanding of societal norms and behaviors. It was found that individuals underestimated the extent to which others were complying with social distancing guidelines, potentially leading to lower adherence to such guidelines themselves (Cookson et al., 2021). This underestimation of compliance with health guidelines among peers illustrates the broader impact of misperceptions on individual and collective behaviors, often leading to suboptimal outcomes.

How misperceptions are “corrected”?
Building on the understanding of misperceptions across various domains, the Social Norms Approach (SNA) emerges as a strategic intervention designed to recalibrate these misperceptions, particularly those related to normative behaviors—actions deemed acceptable or typical within a community (Berkowitz, 2009; Cislaghi & Berkowitz, 2021). By providing accurate reflections of the behaviors and attitudes that prevail within a group, SNA aims to realign individual actions with these collective realities, thereby fostering healthier and more socially responsible practices. For example, in the context of mitigating alcohol misuse among college students, SNA interventions have strategically disseminated data on the actual, often less excessive, drinking behaviors of their peers, with the goal of cultivating healthier drinking habits(Perkins & Berkowitz, 1986; Ridout & Campbell, 2014). Similarly, to bolster pro-environmental behaviors, SNA has been utilized to correct overestimations of peers' environmentally detrimental practices, thus promoting a more accurate perception of community norms regarding environmental stewardship(Huber et al., 2018). 
However, the application of SNA comes with its own set of challenges and ethical considerations that warrant careful examination. The approach's success is heavily dependent on the accurate collection and interpretation of normative behavior data, a task that is inherently complex due to hard-to-measure norms (Burchell et al., 2013; Dempsey et al., 2018). This complexity raises concerns about the potential for misinterpretation of data, which could inadvertently normalize undesirable behaviors if SNA interventions are not meticulously designed and implemented. Moreover, the focus on adjusting individual behaviors to align with perceived group norms may overlook the rich tapestry of individual motivations and the impact of cultural diversity on behavior. This oversight could limit the effectiveness of SNA interventions across different demographic and cultural contexts. These considerations underscore the importance of a deep and nuanced understanding of social norms and the contexts in which they function. As such, when designing and implementing SNA-based interventions, it is crucial to navigate these challenges with precision and sensitivity, ensuring that the interventions are both effective and ethically sound.
Exploring the complex interplay between social learning and the spread of cultural norms, this research project seeks to deepen our understanding of the dynamics of misperception and its impact on individual actions. Specifically, the study aims to unravel how different social learning strategies, demographic and socioeconomic variables, and social group sizes contribute to the formation of misperceptions. This exploration is pivotal, as existing literature predominantly demonstrates the impact of misperceptions on behaviors without dissecting the underlying mechanisms that facilitate this process. By shedding light on these mechanisms, the project aims to generate valuable insights that could inform the design of social norm interventions, particularly for individuals more susceptible to misperceptions. Furthermore, the research will address the methodological challenges associated with data collection and the measurement of intervention effectiveness through the innovative use of agent-based modeling. This approach is instrumental in simulating the development of misperceptions and assessing the potential impact of interventions before embarking on empirical research. To ground this investigation in a concrete and culturally relevant context, the project will focus on parenting norms among U.S. college student populations. This specific demographic group offers a unique opportunity to examine how misperceptions influence behaviors within a well-defined cultural setting.

Hypotheses
The landscape of parental investment in the United States has seen a marked evolution over recent decades, characterized by a multifaceted approach that encompasses time, money, and adherence to intensive parenting norms. A significant trend observed is the increasing financial investment in children, with parents allocating larger shares of their income towards their offspring's upbringing, education, and extracurricular activities, reflecting heightened aspirations for their children's future success and well-being (Kornrich, 2016a). Concurrently, there has been a notable shift in the time investment by parents, with an uptick in the amount of time spent engaging in activities that promote the physical, cognitive, and emotional development of their children (Dotti Sani & Treas, 2016; Gimenez-Nadal & Molina, 2013). This shift towards more time-intensive parenting practices is indicative of a broader societal move towards what is often termed 'intensive parenting,' where the quality and quantity of time spent with children are seen as pivotal to their development (Lee, 2023). Moreover, the norms surrounding parental investment have also evolved, with a growing emphasis on ensuring that children not only achieve academically but also acquire a broad set of skills and experiences that are believed to be essential in today's competitive world. Hence, we first hypothesize that our findings will corroborate the increasing trend of parental investment in the U.S. population, especially in the time parents dedicate to their children's development. This expectation is grounded in the observed escalation of time spent on children's education and activities, alongside a notable rise in parental engagement in child-rearing practices that promote physical, cognitive, and emotional growth.
Given this evolving landscape, it is imperative to examine how these changing norms are perceived across different societal segments, particularly among students. This demographic, often not yet engaged in parenting themselves, derives their understanding of parental investment from indirect sources such as media, societal narratives, and observations of family and peers. This indirect exposure may skew their perception, leading to potential misinterpretations of the nature and extent of parental investments. In an era where 'intensive parenting' has become increasingly normative, there is a tendency for these efforts to be either idealized or overstated by those not directly involved in parenting activities. This phenomenon is supported by findings from studies on intensive parenting norms and their societal impact. For instance, research indicates that intensive mothering expectations, which are deeply ingrained in societal norms, often set unrealistic standards that can influence perceptions even among those not engaged in parenting (Forbes et al., 2020). Additionally, the visibility of intensive parenting in both social and traditional media, coupled with societal narratives that underscore the critical role of extensive parental involvement for child success, may contribute to these exaggerated perceptions (Chin & Phillips, 2004). Consequently, we hypothesize that U.S. students are likely to overestimate the current level of parental investment, particularly regarding the time parents dedicate to their children's development. This hypothesis is informed by the recognition that while intensive parenting practices are widely discussed and aspired to, their actual implementation may not be as pervasive as perceived by individuals outside the immediate family context, leading to potential misconceptions among students about the average parental time investment in the U.S. population.
Lastly, the intricate dynamics of social learning strategies and the size of social groups play a pivotal role in shaping the misperceptions among students. If an individual's learning is heavily influenced by conformity bias, they are likely to adopt the most prevalent behaviors observed within their social circle, eschewing independent analysis in favor of mirroring the dominant parenting norms of their group (Kendal et al., 2018; Laland, 2004). This phenomenon underscores the significant impact of social learning strategies, where the frequency of certain behaviors within a group can dictate the norms that an individual chooses to adopt. Moreover, the application of these social learning strategies is intricately linked to individual demographic and socioeconomic variables. For instance, the educational gradient in parental time investment sheds light on the potential for class diffusion, suggesting that individuals with higher education levels may wield more influence in terms of norm spreading (Dotti Sani & Treas, 2016; Kornrich, 2016b). This interplay between social learning strategies, group size, and individual characteristics highlights a complex web of factors that contribute to the formation of misperceptions. Therefore, we hypothesize that students with larger social group sizes and those from lower household incomes and educational backgrounds are more susceptible to misperceptions. This hypothesis aims to unravel the nuanced ways in which social learning strategies, demographic factors, and socioeconomic variables converge to influence students' perceptions of parental investment norms.
